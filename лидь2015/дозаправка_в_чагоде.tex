\chapter{Дозаправка в Чагоде. 08.08.15}
%\corner{64}
\vepsianrose

Фокус с сосновым и еловым лапником под палаткой удался\mdash я не замёрз, хоть ночь и утро были холодными. С уверенностью могу сказать\mdash можно брать этот способ на вооружение. Утро прошло уже привычно\mdash приготовили овсянку с сухофруктами, позавтракали. Действия моей бригады становятся более слаженными\mdash нам удаётся достаточно быстро свернуть лагерь. Лапник, лежавший под палаткой, решил оставить как есть\mdash кто\sdash то после нас, возможно, использует его как растопку костра. Спустили~байдарку на воду и уже привычным  способом равномерно разместили вещи по отсекам. 

В этот день должны были попасть на реку Чагодощу и~дозаправиться продуктами в посёлке Чагода, который стоит у слияния двух рек\mdash Чагоды и Песи в Чагодощу. На воду встали нормально\mdash не поздно. Вскоре, после очередного поворота реки и парочки островков в русле, впереди показались огромные быки бывшего моста старой дороги Чагода\thinspace\nobreakdash---\thinspace Сазоново. После~них мы прошли мимо ещё парочки островков, отмечая себе, что они никак не соответствуют некоторым описаниям в туристических отчетах, что, мол, на~них вполне себе можно остановиться на ночлег\mdash это~не~так. Островки все заросшие ивняком и травой и~подходят разве что только для очень экстренной стоянки. Впрочем, всё зависит от уровня воды. В~этот маловодный в~здешних краях год всё выглядело так, как нам показалось, что не исключает иной картины при другом уровне воды. 

Уже к полудню прошли железнодорожный мост\mdash ориентир, после которого ещё метров 100 и Чагодоща. Места красоты неописуемой! Опасался, что ближе к устью река сильно обмелеет. Так и вышло. Дно песчаное, шли нормально, сели на мель всего 1\thinspace\nobreakdash--\thinspace 2 раза, да и то быстро выходили на глубину, отталкиваясь веслами ото дна. Река~резко стала шире, видны редкие домики на обоих берегах. Несколько раз проплывали под ЛЭП. 

Из отчётов по маршруту я знал, что где\sdash то в Чагоде на левом берегу есть старый семафор и его видно с~реки. Нашли~это место, причалили\mdash не сфотографировать семафор я просто не имел права\mdash когда и где ещё такое увидишь? Координаты\mdash \CoordsChagodoschaSemaphor. Семафор, правда, был уже без признаков жизни. Рядом два железнодорожных пути\mdash один видно, что заброшен\mdash ржавые рельсы, деревянные шпалы, запах дёгтя. А второй путь очень даже действующий. До железнодорожной станции Чагода не пошли\mdash там около километра по путям$\ldots$ а зря! Могли бы узнать расписание местных поездов, <<кукушек>> как их называют, если они вообще ещё остались. 
\newpage
Потом проплыли местный пляжик на правом берегу и~спросили у~отдыхающих как лучше пройти к магазинам. Нам~дали ориентир\mdash автомобильный мост через реку. Его~ни с чем не перепутаешь. К магазинам\mdash на левый берег у~моста, сказали местные. Это~совпадало с прочитанным ранее в отчетах, и мы стали искать, где бы причалить у~левого берега. 

Мимо~нас прошла моторная лодка с рыбаком. Мужик~мудро погасил скорость заранее, убрав губительную для нас волну\mdash я уже думал разворачивать байдарку поперёк реки, носом на волны, чтобы нас не кильнуло ненароком боковой качкой. Но~всё~обошлось. 

Вскоре увидели обещанный автомобильный мост и перед ним что\sdash то вроде небольшого деревянного причала или мостка у левого берега. Координаты\mdash \CoordsChagodoschaFifteenGoToStore. Причалили,~вытащили коврики и раскладной стульчик. Стали~собираться в~город. Накинул~штормовку\mdash надо выглядеть прилично! Лёню~оставили сторожить байдарку, а~мы с~Аней, взяв сумки, пошли пробираться к магазинам\mdash поднялись по~крутому подъёму к чьему\sdash то забору и пошли вдоль него по тропинке. 

Выбравшись из частной застройки на дорогу и перейдя её, вышли к промзоне местного стекольного завода. Куда~дальше\mdash не~ясно. Вернулись~слегка назад, к двухэтажному дому по адресу Высоцкого 71. Я~подумал сначала, что это в честь Владимира Семёновича. Однако~на~стене дома была закреплена табличка, в которой было сказано, что улица названа в честь Высоцкого Кузьмы Демидовича\mdash участника советско\sdash финской войны, Героя Советского Союза. На момент смерти от ранений в госпитале в 1940 году ему было 29~лет$\ldots$ 

Спросили дорогу у женщины, работавшей в~палисаднике этого~дома и, спустя минут 5\thinspace--\thinspace 10 оказались, неожиданно для себя, в центре посёлка. Жара стоит нешуточная, на небе плывут редкие облачка. В штормовке становится жарковато. 

Находим продуктовые, а~также, что особенно приятно, пекарню. Пополняем запасы продовольствия, а~свежий хлеб берём в пекарне, где, кроме всего прочего, продают вкуснейшую пиццу\mdash её тоже забираем себе сейчас на~перекус. Потом, спросив дорогу, идём искать мясной рыночек. Там оказывается закрыто, хоть сегодня и~суббота. Аня не отчаивается и отправляется за мясом в~другой продуктовый. Я~пока жду её на улице. Возвращается с мясом и~скумбрией. Свежее мясо\mdash это шикарно, а будет ещё и рыба\mdash разнообразие. А то Лёня никак нас уловом что\sdash то не~радует. Аня пока стоит с сумками и кушает мороженое, а~я~быстренько пробегаюсь по главной улице до~площади\mdash хочется посмотреть посёлок.

Из капитальных построек\mdash здание милиции, почты, сбербанка. Несколько магазинов продуктовых и~хозяйственных. Гостиница ещё и~автостанция. Всё~это сосредоточено на главной улице, метров 500 в длину, под~названием Кооперативная. Остальное\mdash деревянные, на~манер бараков, старые двухэтажные жилые дома. Выхожу~на~площадь\mdash пустовато. Видна новая деревянная церковь слева в отдалении. Слева же в глубине от~площади через сквер\mdash здание администрации. В начале Кооперативной улицы\mdash памятник павшим в Великой Отечественной Войне. Не могу пропустить такое\mdash иду туда. 

4 уроженца Чагоды были удостоены звания Героя Советского Союза\mdash стоят памятные стелы. Вечный~огонь. Снимаю панаму. Суровый памятник и имена павших. Нахожу двоих однофамильцев$\ldots$~постоял, проникся моментом. Место~среди высоких голубых елей навевает какую\sdash то строгость и служит напоминанием о тех страшных днях, о которых всё реже вспоминают$\ldots$~а~есть о чём! Это сейчас Савёловская железнодорожная ветка от Москвы на Санкт\sdash Петербург, а тогда Ленинград, находится в полузаброшенном состоянии и по ней идут в~основном товарняки, да пригородные электрички местами. А~в~тяжелые дни Великой Отечественной она была одной из~основных, после того как в 1941 году Октябрьская железная дорога оказалась повреждена, и отдельные её~участки к~1942~году были заняты немецко\sdash фашисткими~захватчиками.

Ветка Москва\thinspace\nobreakdash---\thinspace Калязин\thinspace\nobreakdash---\thinspace Овнище\thinspace\nobreakdash---\thinspace Хвойная\thinspace\nobreakdash---\thinspace Мга была достроена и замкнута к 1918 году, образовав резервный путь на Ленинград. Во время войны на~неё, а~также вологодскую ветку, легла колоссальная нагрузка по снабжению Ленинградского фронта. В~спешном порядке велось строительство окружных и радиальных железнодорожных путей в ленинградской области. Так~появились ветки Кабожа\thinspace\nobreakdash---\thinspace Чагода и Подборовье\thinspace\nobreakdash---\thinspace Чагода, причём строилось всё в рекордно короткие сроки. В~посёлке Хвойная, что на участке пути Неболчи\thinspace\nobreakdash---\thinspace Кабожа, располагался один из штабов Ленинградского фронта, аэродром, а также госпиталь. Обязательно надо побывать в Хвойной, решаю я, стоя перед стелами с бесконечными списками фамилий$\ldots$

Через Чагоду шло сообщение вологодской ветки железной дороги с савёловской, обеспечивая снабжение ленинградского района. В Заборье, где был наш стапель, начиналась знаменитая Дорога Жизни к Ленинграду с~начала ноября до конца декабря 1941 года во время боёв за Тихвин. Я всё ещё стою около памятника, не~в~силах уйти. По~спине проходит какой\sdash то неведомый холодок. Перед глазами встают картины эшелонов, идущих мимо того семафора, который мы сегодня проплывали, колонны техники$\ldots$~суматоха и толчея, налёты вражеской авиации$\ldots$  

Очнувшись, понимаю, что надо идти обратно\mdash Аня наверно уже заждалась. С мыслью, что непременно надо побывать в Хвойной, иду бодрым шагом назад. Заброситься, правда, до Хвойной от Москвы поездом уже не удастся\mdash к началу 2000\sdash х годов отменили последний пассажирский поезд сообщением Москва\thinspace\nobreakdash---\thinspace Санкт\sdash Петербург, ходивший по савёловской ветке. Остаётся~вариант через Санкт\sdash Петербург\mdash оттуда поездом до Сонково. Через~Хвойную протекает речка Песь, которая, как уже говорилось, также впадает в~Чагодощу. Вот~так и сложился новый вариант байдарочного похода на~следующий, надеюсь, год$\ldots$~время расставит всё на~свои~места.

Вернувшись к магазину, забираю две сумки с~продуктами и мы с~Аней начинаем обратный путь, срезав вдоль забора стекольного завода, через проходную которого видны огромные тюки с чем\sdash то непонятным\mdash то~ли готовая продукция, то~ли сырьё. Переходим дорогу у автомобильного моста, снова пробираемся через дачный посёлок и находим Лёню, спящего на берегу под палящим солнышком. Укладываем продукты компактнее в байдарку, перекусываем той самой пиццей, купленной в пекарне, и~кефиром. Небольшой отдых и спустя минут 20 мы готовы. Отчаливаем. Без проблем проходим автомобильный мост в~среднем пролёте и гребём~дальше. Погода\mdash блеск! Ветерок и жаркое солнце, на~небе немного облачков. 
\newpage
Постоянно смачиваю панаму забортной водой\mdash высыхает в~мгновенье~ока. Ребята держат курс, а я сверяюсь с картой~и~GPS. Прикидываю сколько мы физически сможем ещё сегодня пройти. Намечается отставание от графика из\sdash за резкого снижения скорости течения\mdash Чагодоща заметно медленнее Лиди. Но это не критично, думается мне\mdash гребём как получается, за рекордом сегодня не рвёмся. Греют~душу сумки с продуктами из Чагоды. Одну из них поставили у~меня в~ногах, и~сидеть стало не слишком комфортно, но~терпимо\mdash на стоянке разложим и~перепакуем всё~как~надо. 

Вечером проходим село Мегрино с красивейшей церковью на левом берегу. Там же два троса поперёк реки\mdash один сигнальный с флажками, другой силовой для парома. Самого парома что\sdash то не видно, возможно лежит в высокой прибрежной траве. 

Догоняем надувную байдарку с двумя колоритными дедами\mdash они идут с~Песи и~встают на~ночёвку около Мегрино. Рассказываю про наше <<блестящее>> прохождение перекатов на Лиди и 15~пробоин, а~они говорят, что~тоже пробились на~Песи~2~раза. Ещё немного поболтав, мы втроём берёмся за вёсла и~отрываемся от них. Позже, в 2019 году, вновь оказавшись на Чагодоще, я встретил их опять. Простите~меня, достопочтенные научные, судя по всему, сотрудники из~Москвы, за моё тогдашнее определение вас в~<<деды>>. Вы~ещё очень даже ого\sdash го!

До планового километража в этот день не дошли всего 7~километров и причалили на ночёвку за селом Мегрино и Горки на крутом левом берегу у изгиба реки. Могли бы упереться, пройти дальше ещё часок и выполнить норму километража на сегодня, но уже подустали, да и мясо с рыбой хочется приготовить до темноты. В сумке плещется красное, приобретённое в Чагоде\mdash сварим к ужину глинтвейн, для чего у нас припасены лимоны, апельсин и даже палочки~корицы. 
%\newpage

Причаливаем, я взбираюсь на обрыв в~разведку$\ldots$~рай~на~Земле\mdash стоянка отличная! Есть~скамейки, костровище, ровная поляна\mdash идеальное место для палаток, а кроме того\mdash куча поваленных, а~самое главное, сухих берез. Экипаж балагурит\mdash звук с пилорамы в Горках их~смущает. Звук~этот конечно меня тоже смущает, но~не~будут же там всю ночь работать? Оставив пока ребят одних, иду по дороге в разведку\mdash кто\sdash то есть ниже по течению\mdash услышал голоса. Выхожу~к~берегу\mdash там местный с семьёй на машине. Обжигает~в~костре честно стыренные откуда\sdash то медные провода, чтобы избавиться от изоляции и сдать цветмет. Разговорились о том, о сём. На противоположном берегу сидят рыбаки\mdash тоже видна машина. Успокоившись, что мужик нормальный, обожжёт медь, да и уедет скоро, возвращаюсь~к~бригаде. 

Стоянка отличная\mdash и столик есть и сосны красивые. В~сумме все факторы подкупают, и я решаю вставать на~этой поляне. Перетаскиваем вещи, байдарку тоже поднимаем на обрыв, разбиваем лагерь. Координаты\mdash \CoordsChagodoschaMegrino. Можно было бы, конечно, встать за поворотом, там, где мужик медь обжигал или чуть подальше. Но там ни~столика, ни~дров, ни~поляны красивой, зато ровный пляжик. Остановило то, что он костер из покрышек жёг\mdash вонища ужас! И вокруг куча этой гари чёрной и мусора. А~ещё говорят, что туристы мусорят. Не верю! Больше~мусора от самих местных, наплевательски относящихся к природе своего, между прочим, родного края. Ничего, на крутом бережку постоим, ничего с нами не случится.

\newpage
%Вырубаю, как обычно, костровые палки 4 штуки\mdash 3~кола и одну перекладину. Натягиваю между колами верёвку для сушки полотенец и одежды. 
Костёр уже готов, и~картошечка варится! Я быстро устраиваю стирку и купание в реке, а потом принимаю наряд на камбуз доделать ужин. Аня тоже решает быстренько вымыться и спускается к воде. Разговоры рыбаков на~противоположном берегу моментально смолкают\mdash всё их внимание сейчас приковано к ней. Мы~же с Лёней остаёмся у~костра и варим глинтвейн. Надо~приступать к готовке мяса и рыбы. Из аниного рюкзака, будто из шляпы фокусника, появляется решётка для жарки~мяса! Что ещё таит этот рюкзак?! Приматываю веточку к~решётке верёвкой, чтобы не~горячо было брать руками, и~приступаем к готовке.

Через полчаса последние приготовления позади. Скамеечки приходятся как нельзя кстати\mdash из одной организуем стол, вторую используем по назначению, а~я~размещаюсь на своём раскладном стульчике. На~ужин у~нас замечательное мясо на огне, рыба с лимоном, глинтвейн, картошка в мундире и помидоры в натуральном виде с~солью. Красота! Пикник как будто, а не сплав. 

Сидим, вкушаем, обсуждаем сегодняшний день и~как классно сегодня проплыли Чагоду. Читаю стихи про скифов и~азиатов с~раскосыми и~жадными очами$\ldots$~хорошо! На~столе горит свеча, припасённая Аней. Романтика, с~ума сойти. Уже~смеркается, зажигаются первые звёзды. Покушав,~пробирает на песнопение. Наверное,~бригаде нелегко переносить эти музыкальные потуги при~полном, как~я считаю, отсутствии у~меня слуха, но~что делать\mdash душа отчаянно требует петь и~я ей~не~препятствую. 

Закатное солнце зашло далеко за горизонт, и тихая ночь спустилась с небес. Рыбаки, что стояли на противоположном берегу, давно уехали. Мужик, обжигавший медь, тоже. На~пилораме также всё давно стихло. Какая же тишина вокруг! Наступила тёмная ночь, и только наш костер, как маяк мо мгле, посреди поляны, окруженной соснами, горит и~стреляет искрами в высокое августовское небо, бриллиантами рассыпавшееся над нашими головами$\ldots$

Аня уходит спать пораньше, а мы с Лёней располагаемся у костра. Я достаю фотострубцину, ввинчиваю её в полено, креплю сверху фотоаппарат и подключаю дистанционный пульт. Будем снимать астропейзажи на длительной выдержке, насколько это получится. Нахожу пару интересных ракурсов, делаю несколько снимков. Усталость вдруг накатывает волной. Времени~уже около полуночи. Сидим, болтаем тихонько, а больше молчим, подымливаем, разливаем. Расслабляюсь окончательно. Горючее исправно заставляет задуматься о~жизни и~далёких галактиках. Чувствую, что спать сегодня будем все просто мертвецки. На~ночь я также обильно подстелил под палатку лапник\mdash будет мягко~и~тепло$\ldots$
  
Падающих звёзд не увидели, как ни старались. Костёр~постепенно стал угасать и мы, подкинув в него большие гнилушки, чтобы те тлели до утра, расползлись по~своим~палаткам. Накачав коврик и переодевшись, я~вдруг вспомнил, что не открывал сегодня судовой журнал. Экое~упущение! Сделав над собой усилие, включил фонарик, собрался с мыслями и черкнул несколько ярких строк про~сегодняшний день. Времени было уже около двух ночи. Глаза просто слипались, и я завалился спать, моментально провалившись, как и обыкновенно в этом сплаве, в спокойный сон на свежайшем и чистейшем воздухе сосновых вологодских лесов.

\begin{center}
	\psvectorian[scale=0.4]{88} % Красивый вензелёк :)
\end{center}
