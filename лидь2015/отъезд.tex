\chapter{Отъезд. 03.08.15}
\corner{64}

И вот, наконец, настал день отъезда. Как я ждал этого момента! Мой первый по-настоящему самостоятельный большой поход. Этим я грезил примерно с 6\sdash го класса, как только начался курс физической географии\ldots~и теперь это\mdash реальность! Это~шаг куда\sdash то вперёд, это абсолютно новый опыт в моей жизни\mdash ничего подобного до этого я не совершал\mdash вот так, полностью автономно и в отрыве ото всего и вся. Я с упоением воображал по топографической карте реку, обрывистые берега, чистейшие сосновые леса и песчаные пляжи\ldots 

Поезд Москва\nobreakdash--Череповец отходил от Ярославского вокзала в 21:05. Заблаговременно, часа за полтора, как всегда не рассчитав со временем, я оказался на вокзале. Совершенно официальная парковка у вокзала обошлась в 100 руб/час. Заняли парковочное место и стали ожидать остальную часть команды. Погода в Москве хмурилась, я кутался в штормовку и одновременно старался как можно меньше сомневаться в успехе предстоящего мероприятия. Небо густо заволокло облаками. Перед парковкой угрюмым от облачности серым цветом расположилась стена Ярославского вокзала\ldots

Ярославский вокзал из Трёх Вокзалов занимает в моём рейтинге 2 строчку\mdash интересная архитектура и особенная красота, по сравнению с другими. Ленинградский по соседству\mdash первую, естественно, поскольку сразу приковывает взгляд строгостью стиля и классическими формами. Про Казанский, это аляповатое уродское сборище из бесчисленного количества непонятных пристроек, надстроек и подземелий, говорить совсем не приходится.

Время ожидания тянулось медленно. За полчаса до отправления или даже раньше подали поезд, и вскоре вынырнула из метро моя команда. Увидав издалека их огроменные рюкзаки, я чуть не выругался в голос\mdash ЧТО они везут?! И это же только одежда и палатка\ldots~без продуктов! Не для них ли я сто раз переписывал и сокращал список вещей?! В первый раз все этим болеют\mdash тащат с собой кучу ненужного: лишние ботинки, резиновые сапоги, лишнюю смену одежды, лишнее ещё\sdash что\sdash нибудь\sdash тяжелое\sdash и\sdash объёмное и прочее\sdash прочее\sdash прочее. Я~хотел уберечь ребят от этого печального опыта, но не сумел\mdash видимо каждый, как ни крути, должен набить себе эту шишку сам, увы. В свой первый сплав я тоже набрал кучу ненужного, чего вполне мог бы не брать. И это было учтено в последующие разы. А тут\ldots видимо, придётся с этим смириться, с этими огромными рюкзаками, что ещё поделать?

Билет на байдарку (какая издёвка от Российских Железных Дорог\mdash подавитесь своими 200 рублями) был приобретён заранее и с посадкой в поезд проблем не возникло. Билеты на экипаж я купил электронные\mdash предъявляешь только паспорт при посадке и проходишь\ldots XXI век в действии, с ума сойти! Отмечу, что фирменный поезд Москва\nobreakdash--Череповец номер 126Я <<Шексна>>\mdash \textit{единственный} поезд из Москвы до Череповца на 2015 год. <<Да~здравствует>> РЖД!
 
В Череповце нас ждёт пересадка на Архангельский поезд, следующий до Санкт\sdash Петербурга под номером 009С. Сходить нам на станции Заборье, как уже неоднократно упоминалось. До~Череповца от~Москвы ехать 8 часов примерно, от Череповца до Заборья около трёх с половиной. Перерыв между поездами 4 часа\mdash за это время как раз надеялись закупиться продуктами. Ехали везде, естественно, плацкартом\mdash плацкарт наше всё, будь ближе к народу! 

Мы подошли к нашему вагону и начали грузиться. Таскали вещи при загрузке мы, конечно, неприлично долго, но внутри вагона довольно быстро раскидали всё по просторам третьих полок, а байдарку положили вниз под сиденье. Место на 4\sdash го члена экипажа, билет которого пришлось сдать, никто не занял, и мы с радостью его оккупировали, благо со стороны проводницы никаких претензий не было.

Жена безумно волновалась за меня, и расставание с ней далось особенно тяжело, хотя Адмирал сплава и не должен показывать виду. И вот, последние объятия, все расселись по местам, закончена посадка. Поезд неслышно трогается с места и плавно набирает ход. Все машут руками, а жена отходит в сторону на платформе, чтобы дольше было видно меня из окна\mdash машу ей рукой в ответ и шлю воздушный поцелуй. Не горюй, милая, я скоро вернусь, а сейчас Дух Авантюризма зовёт меня вдаль!

Платформа вскоре скрылась из виду, и мы продолжили размещаться в вагоне. Этот самый единственный поезд до Череповца, хорошо ещё, что фирменный, оснащённый биотуалетами и прочими благами цивилизации, удобно прибывает\mdash утром. Как раз ночку поспать и уже приедем. В первый раз мне довелось ехать в плацкарте на нижней полке\mdash всегда брал верхние до этого, куда бы ни ехал. Удобно, что сказать. Никто не сшибает головой твои ноги, свисающие в проход со второй полки при росте за 190 сантиметров. Теперь, на нижней полке, мои ноги сшибают ногами, а не головами. Но не сильно и не матерясь. Терпимо. 

Поезд, тем временем, уже выехал из Москвы, и начались обычные плацкартные разговоры.  Обо всём по чуть\sdash чуть. Плацкарт, как явление, вещь незаменимая и ни с чем несравнимая. Если РЖД его когда-нибудь отменит\mdash это повод для новой Великой Октябрьской как минимум, ибо это будет чересчур. Плацкарт\mdash это не просто способ максимально дёшево добраться куда бы то ни было, это\ldots своя атмосфера. Это полки, это проход, это титан в начале вагона, это еда из дома в фольге, а ещё ноги, свисающие в проход\ldots ну, словом, вы понимаете. А если не понимаете, срочно проверьте свой паспорт\mdash вы точно гражданин России? 

Взяли чаю по 30 что ли рублей. Очарование чая в стакане с подстаканником и постоянно позвякивающая ложечка\ldots иностранцам не понять\mdash им главное комфорт, выгода, деньги. Тут же иное. Долго болтали о разном, шутили. Волнение команды было видно невооружённым глазом, да я и сам, естественно, волновался, но романтика путешествия и жажда неизведанного звала вперёд. И всё\sdash таки, в какую авантюру мы ввязываемся! Ух!!! Я был просто горд собой, что решился на такое и сумел, конечно не без поддержки начальника, организовать свой первый по\sdash настоящему дикий сплав. 

Народ вокруг стал укладываться и мы, последовав их примеру, тоже улеглись пытаться заснуть под стук колёс. Места у нас были в купейной части плацкартного вагона\mdash два нижних и одно верхнее. Я расположился внизу, а Лёня с Аней на двух полках напротив. Вскоре выключили основное освещение, оставив только приглушённое ночное дежурное. Поезд смиренно отмерял километры ярославской железной дороги стуком колесных пар\ldots

Удалось устроиться удобно и я лежал, укутавшись в простынь, пытаясь заснуть. Но сон не шёл\mdash я не сильно устал за день и вечер, к тому же перекидывался смс\sdash ками с женой, ловя моменты, когда была связь. Жалко, что она не идёт со мной в этот раз и безумно грустно от этого\ldots уж кто\sdash кто, а она\sdash то идеальный напарник в походе, в чем я не раз убеждался.
 
Отгоняю от себя вдруг налетевшую тоску мыслью о том, что я\mdash Адмирал сплава и двое зелёных матросов, уже храпящие к этому моменту на соседних полках, на реке будут полностью зависеть от меня. Мне, как предводителю, Адмиралу, придётся давать указания, держать всё под контролем, при этом стараясь не переусердствовать, и не забывать самому принимать в нашем действе самое активное участие. Неведомое ранее чувство особенной ответственности подкатывает, но потом я вспоминаю книжку <<Географ глобус пропил>> (а также недавно вышедшее по ней кино) и меня отпускает. Одновременно возникает странное неизведанное чувство отрешённости от мира\mdash как будто вся привычная жизнь идёт параллельно нам, а мы как бы со стороны, на обочине её, в параллельных мирах; как будто мы\mdash не мы, а кто\sdash то другие, в тельняшках и штормовках\mdash сели в поезд и несёмся навстречу неизвестности. Вскоре веки мои тяжелеют, и я проваливаюсь в сон.

\begin{center}
	\psvectorian[scale=0.4]{88} % Красивый вензелёк :)
\end{center}