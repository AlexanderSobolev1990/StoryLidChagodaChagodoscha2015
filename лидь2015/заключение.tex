\chapter{Заключение и выводы}
\corner{64}

А какие могут быть выводы? Понравилось? Однозначно! Ещё~пойду? Несомненно! Это, пожалуй, главное, что можно сказать. Я добился своего\mdash осуществил свою мечту\mdash сходил в самый настоящий поход, который организовал сам. Все невзгоды, вроде пробоин или ломоты в мышцах после пройденного километража со временем забудутся, останутся только самые яркие и светлые воспоминания\mdash <<что прошло, то будет мило>>. И это действительно так\mdash вспоминаю сейчас свои прошлые сплавы\mdash по Зилиму, Белой. Уже не имеют значения ни~ломота в руках, ни то, что промок под дождём или от~брызг с~весла. Помнишь только ощущение бесконечного простора, вкус пьянящей свободы, чувство единения с~природой, дым костра, атмосферу похода\ldots 

Чувства, с которыми я вернулся из этого сплава просто не изложить на бумаге. Возможно впервые в жизни я почувствовал в себе не просто некую уверенность, а именно \textit{силу}. Силу человека\sdash первопроходца, путешественника, первооткрывателя. Эмоциональный подъём был колоссальный. Вероятно это чувствовали варяги, викинги, вылезая из своих драккаров по возвращению домой из удачного похода\ldots~Словом, я правда вернулся другим человеком. Человеком, вкусившим чувство безграничной первобытной свободы, отсылающее в самую глубину веков\ldots

Выводы, в целом, по нашему сплаву можно сделать такие:
\begin{enumerate}
	\setlength{\itemindent}{-1em}
	\item По маршруту:
	\begin{enumerate}
		\setlength{\itemindent}{0em}
		\item [$-$] Лидь неширокая, красивая, местами даже быстрая в августе речка. Имеются, вопреки отчетам, перекаты и шиверы на участке от Заборья до~устья. При неумелом прохождении шивер и~малой воде\mdash пробиться очень легко. Шкура для данного маршрута должна быть с хорошими леями однозначно. Под кильсон и штевни лучше подкладывать куски туристического коврика – хуже не будет точно. Заманчиво начинать с~верховьев Лиди, но для этого требуется заброска автотранспортом в район д.~Радогощь.
		\item [$-$] Чагода\mdash слишком мал участок её, по которому мы шли, чтобы оценить реку в полном объёме. Она, несомненно, шире, чем Лидь, течение замедляется, чаще начинают попадаться островки посреди реки. Вставать на них с ночёвкой, как пишут в некоторых отчётах, можно с большой натяжкой. При малой воде в устье Лиди и сразу после него в Чагоде имеются перекаты.
		\item [$-$] Чагодоща\mdash красавица река! Широкая, местами мелкая, быстрая под берегами. Характер берегов после Залозно изумительный, с~шикарным меандрированием. Высокие песчаные обрывы с~сосновыми борами наверху и протяженные мели с кустарником никого не оставят равнодушным! Cтоянок достаточно, есть и со столиками, скамьями.
	\end{enumerate}
	\item По заброске/высброске:
		\begin{enumerate}
		\setlength{\itemindent}{0em}
		\item [$-$] Заранее узнавать телефоны водителей, которые помогут со стапелем/антистапелем, если это требуется. Однако, как показывает практика, можно найти всё и на месте.
		\item [$-$] Байдарку по новым правилам теперь можно провозить в поезде совершенно легально, заплатив как за багаж весом 30 кг (по состоянию на 2015 год). При возможности, лучше разделить упаковку с байдаркой на 2 части\mdash так будет проще размещатся в поезде и другом транспорте.
		\item [$-$] Закладывать больше времени на стапель и~антистапель\mdash километраж на эти дни планировать минимальный.  
	\end{enumerate}
	\item По экипажам:
		\begin{enumerate}
		\setlength{\itemindent}{0em}
		\item [$-$] С особой тщательностью следует подбирать экипажи и только в самых крайних случаях идти с~малознакомыми людьми. Про незнакомых я вообще молчу\mdash этого, по возможности, следует избегать. Исключение\mdash всякие коммерческие сплавы.
		\item [$-$] На трёхместных байдарках физически сильных гребцов следует размещать на корме и носу\mdash так проще управлять байдаркой и совершать маневрирование. 
	\end{enumerate}
	\item По снаряжению:
	\begin{enumerate}
		\setlength{\itemindent}{0em}
		\item [$-$] Клей и заплатки\mdash брать! Конечно, это был мой первый сплав на байдарке и я толком не знал её <<предела прочности>>, но всё равно\mdash в случае чего ремкомплект должен быть, что называется, <<на все случаи жизни>>\mdash помимо заплаток и клея можно взять гвозди, проволоку, верёвки и т.д.
		\item [$-$] На случай встречи с дикими зверями лучше, конечно, иметь в команде охотника, но наиболее распространённый способ решения данного вопроса\mdash фальшфейеры и патроны <<сигнал охотника>>.
		\item [$-$] Лишние вещи\mdash будут, причём как свои, так и~других членов экипажа. Стоит относится к этому, как к неизбежности.
		\item [$-$] Равномерно распределять групповое снаряжение между членами как одного экипажа, так и между экипажами, если это требуется (котелки, топоры).
		\item [$-$] Снаряжение, обувь, одежда, палатка\mdash хорошо, если это всё не новое и не дорогое\mdash не жалко порвать о сучья при заготовке дров, испачкать в сосновой смоле или получить прожжённую дырочку от искры~костра.
	\end{enumerate}
\end{enumerate}

На этом позволю себе закончить повествование о замечательном и увлекательном, полном экстрима приключении на реках Лидь, Чагода, Чагодоща. Я~честно постарался изложить как можно подробнее не только наш маршрут и места стоянок, но и свои личные впечатления и переживания. Сплав, без всяких сомнений, доставил мне и экипажу массу удовольствия. Мы~побывали вдали от~цивилизации, отвлеклись от городской суеты и обыденности. Это имеет огромную цену! Кроме того, оказавшись дома, я понял, что открыл своё место силы, свой источник жизни, свой край\mdash край Верхней Волги.

С надеждой и рвением смотрю я на рюкзак с байдаркой, лежащий на чердаке. На какую реку пойдем в следующем году? Песь?
\begin{center}
	\psvectorian[scale=0.4]{88} % Красивый вензелёк :)
\end{center}

\begin{center}
	\Large {КОНЕЦ}
\end{center}
\vspace{\fill}
\begin{flushright}
	\copyright~Соболев~А.А.,~Москва,~27.08.2015\\
	\textit{ред.~17.11.2019,~18.06.2022}
\end{flushright}



