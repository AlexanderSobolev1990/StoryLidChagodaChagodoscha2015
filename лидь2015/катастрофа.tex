\chapter{Катастрофа. 05.08.15}
\corner{64}

Проснулся от того, что замёрз. Холод шёл от земли к спине. Никак не ожидал такого развития событий\mdash мы сплавлялись с~женой на Южном Урале в августе и~отнюдь не~мёрзли, хотя спали почти всегда на каменистых берегах рек, да и на земле тоже. И~спальник был тот же, и~туристический коврик такой же. Правда,~там мы находились южнее градусов на 5 по широте. Неприятно, но~что делать. Надо будет что\sdash то придумать, мёрзнуть так не годится. 

Встал, размялся, вышел по дороге в поле\mdash жидкий рассветный туман, солнышко только\sdash только встало. Красиво\ldots~ради таких моментов стоит жить. Стою~в~каком\sdash то оцепенении, встречаю рассветное солнце. Сосновые леса кругом, воздух чистейший, в дымке встаёт оранжевое солнце\ldots~картина заворожит любого. Я стою и взгляд мой спокоен, в голове наступает странный покой и порядок, мысли упорядочиваются и угасают\mdash ничто не тревожит, ничто не отвлекает на ненужное, лишнее. Тут~вообще ничего нет лишнего\mdash только человек и природа. Так~и~стою\ldots~долго стою, пропитываясь белым рассветным туманом.

Но довольно медитации, пора в лагерь! Натаскал дровишек, развёл огонь и стал готовить кашу на завтрак. Бригаду, естественно, тоже поднял ни свет ни заря\mdash полез за гермомешком с провизией, который лежал у них в палатке. Ничего, потом отоспимся\mdash надо на воду, быстрее! Жажда приключений! Нас~ждёт красавица Лидь! 

Паковались опять очень долго. Выясняются неприятные подробности, что мой экипаж тащит с собой бытовой фильтр\sdash кувшин для воды и абсолютно всю воду, которую мы употребляем, желает пропускать ещё и через фильтр, прежде чем кипятить. Более того, зубы, оказывается, моя бригада тоже чистит исключительно фильтрованной водой. Меня пробирает на крепкие выражения. Такого ... я вообще не слышал. Чтобы~чистейшую воду, не~прошедшую ещё ни одного крупного города или завода, воду, в~которой плещется вечерами рыба и~лютуют выдры\mdash показатель чистоты реки\mdash и ещё её пропускать через фильтр?! Это~долго и, более того, не нужно\mdash кипячения более чем достаточно. Готовка еды и чая из\sdash за этого фильтрования сильно затягивается, что нервирует меня нешуточно. Коричневатый~оттенок воды\mdash это от торфа и~песка, ничего страшного! Как не хватает в такие моменты нашего 4\sdash го человека, который не смог с нами пойти. Он\sdash то уж точно не позволил такой ереси сбыться. 

Потом показываются из рюкзака резиновые сапоги нашей Ани\ldots~для кого брались лёгкие и компактные бахилы химзащиты, выполняющие роль сапог? Зачем тащить эту резиновую тяжесть, занимающую к тому же много места? Таких непоняток встретится мне немало. Это\mdash плата за~несхоженность и~неопытность экипажа. Я оказался на~реке во главе людей, которых видел фактически 4\sdash й раз к~моменту стапеля. И хотя я целый вечер тогда ещё в Москве у себя на~кухне распинался им про все тонкости и~особенности водного похода, сплава, а~потом ещё и написал инструкцию что брать с собой, всё равно вышло так, как вышло. Возможно, если бы у них был опыт турпоходов, было бы легче, но увы. Люди~оказались морально не готовы к~таким, казалось бы, очень простым вещам, как вода из~реки. А ещё и вещей понабрали\mdash мама дорогая! Надо~было заранее встретиться перед сплавом и~повыкидывать ненужного\ldots~вспоминаю при этом свой первый сплав сразу же. Не скажу, что я тащил много ненужных вещей, но они точно были. Не припомню досконально что именно, но должен признать, что тащил сандалии, хотя обошёлся в итоге и без них, тащил зачем\sdash то второй свитер, несколько футболок\ldots~наверняка ещё что\sdash то. Естественно, во второй сплав я брал минимум вещей, умудрённый прошлым опытом. Но в моём первом сплаве было одно большое НО. Тогда я мог брать с собой хоть бензопилу и слона на верёвочке\mdash катамаран обладает безумной грузоподъемностью\mdash там вещи лежат на настиле между баллонами и никому не мешают. Иное дело байдарка. Разве самому приятно сидеть всему обложенному тюками, притом что в <<Таймени>> вообще\sdash то прилично места? 

Ладно, как бы там ни было, на воду мы встали почти вовремя и бойко полопатили вёслами дальше. Я старался не показывать излишнего раздражения из\sdash за кошмарного количества вещей и фильтрования воды\mdash авось проблема рассосётся\mdash в конце концов, еды, а значит и объёма груза, будет становиться с каждым днём всё меньше и меньше; с~фильтром решил так\mdash в свой наряд на камбуз я фильтром пользоваться не буду. На этом решил проблему закрыть, чтобы не портить ни себе, ни бригаде впечатлений от отдыха.

Экипаж же мой был в приподнятом настроении после Адмиральской Каши, и мы неслись навстречу приключениям, которые не заставили себя долго ждать. Но~не этого ли мы так жаждали, не к этому ли так стремились? Нам~то и дело встречались завалы, но благо река за конец весны и лето уже расчистила себя путь, и обноситься нам не пришлось. Лишь~временами острые ветки и сучки на~брёвнах, торчащих из воды, доставляли нам неудобства, да~и то чисто психологические. Мне нравился характер реки и то, что мы один на один со стихией. Одни в беспросветной глуши на стыке двух областей, двух миров\mdash обыденного и~параллельного, где~ты свободен от всего\ldots~

Так, скользя по водной глади и увиливая от валунов, мы миновали ЛЭП близ Забелья. Затем впереди показалось раздвоение русла, и мы пошли в правой протоке\mdash там мне показалось глубже. Через несколько сот метров сужение русла и\ldots~зашумел водопадик! Времени отвлекаться на~навигатор не было, да и у меня не осталось сомнений\mdash я вспомнил карту\mdash это остатки плотины бывшей местной ГЭС! Страх и ужас, обуявшие меня поначалу, быстро сменились задором и азартом пройти это препятствие. Причаливать~для осмотра не стали\mdash я выровнял курс байдарки и направил её аккурат на середину слива. Будь~что будет, решил я, штурмуем сходу! Секунда и\ldots~мы преодолели этот перепад! У\sdash у\sdash ух! Быстро заложили крутой поворот влево и затем вправо, следуя изгибу реки и~стараясь держаться середины русла. Прошли мимо неплохой стоянки на правом берегу. Но становиться на ночлег нам ещё очень рано\mdash в планах моих добраться сегодня до~Тургоши и даже немного пройти ниже. 

В этот день прошли устья меленьких речушек\mdash Белой и Межницы\mdash все на правом берегу. Перекатики небольшие встречались нам по пути постоянно и~днищем мы цепляли очень часто\ldots~река начала стремительно ускоряться и приближаться к железнодорожной ветке Подборовье\nobreakdash--Кабожа.

А потом внезапно начался АД. Череда шивер и~перекатов с белой бурлящей пеной и огромными валунами в русле. Как~говорится, вот это поворот! Как хорошо, что шкура пролеена\mdash вся надежда на прочные леи, идущие вдоль стрингеров и под кильсоном. Хоть бы всё обошлось! Я~ни~разу не~проходил серьёзных перекатов и~шивер в своём сплавном опыте. Единственное правило, которое я знал, да которое и так понятно\mdash надо идти по тёмной воде, поскольку там глубже. В~первом нашем сплаве с женой была пара перекатов, но они легко проходились даже на~катамаране. Во~втором сплаве я вообще таких приключений не припомню, а на Киржаче, где мы провели сплав выходного дня, там и вовсе только обнос плотины, да один лёгкий слив. Всё. Тут же начался просто кошмар. 

Временами мы налетали на камни, потому что Капитану с кормы трёшки достаточно плохо видно, что творится впереди. Плюс~в~первые дни сплава я ещё не~догадался подкладывать гермомешок с вещами под себя на манер сиденья и сидел низко, на самом дне байдарки. Соответственно, спины экипажа сильнее закрывали мне обзор. Один раз на перекате байдарку поставило поперёк реки и прижало к валуну, а вода уже вот\sdash вот была готова перелиться через край фальшборта и тогда случилось бы страшное\ldots~быстро перенеся вес на другой борт и резко скомандовав то же самое сделать и экипажу, спасли положение, убрав губительный крен. Вылезли из байдарки, удерживая её от крена, провели на чалке, миновав опасное место, и пошли дальше. Зачерпнули с Лёней на перекате в~химзащиту воды, когда оттаскивали байдарку от валуна. Пришлось снимать химзу, выливать, снова надевать\ldots~и вода на перекате не слишком тёплая, а прямо скажем холодная. 

На этом же перекате я травмировал колено\mdash его зажало набегающим водным потоком между байдаркой и валуном, когда я на~секунду замешкался, вылезая. Тогда, я конечно, не придал особо этому значения, поскольку был, что называется, на~адреналине, но позже эта травма доставляла мне множество неудобств.

Когда садишься в этих бахилах ОЗК в байдарку, как ни отряхивай, вода с них всё равно нальётся. Я полагал, что собранная на дне байдарки вода\mdash с бахил, но позже, анализируя этот день и эпизод на этом перекате, я пришел к выводу, что уже тут мы получили парочку фильтрующих пробоин. Через такую пробоину вода не хлещет фонтаном, а потихонечку сочится. В принципе не страшно, главное периодически отчерпывать её кружечкой.

О кружечке отдельная история. По совету начальника я~прикупил простую белую эмалированную железную кружку. Любовно оплёл её ручку паракордом\mdash получилась красота. В~сплаве эта кружка\mdash то, из чего пить и чай, и что покрепче, и чем отчёрпывать воду со дна байдарки. Так вот, кружка предусмотрительно была положена в карман штормовки и я всегда был готов ей воспользоваться.

Бодро идём вперёд и\ldots~правильно! Снова череда перекатов. Снова какой\sdash то ад. Я уже понимаю, что воды в~этом году мало и к августу Лидь совершенно неприлично обмелела местами. Неудачно входим в створ переката, если так можно выразиться, и нас начинает мотать из стороны в сторону. Налетаем на камень, байдарка кренится\ldots~камни трутся о шкуру, слышен металлический <<Чпок!>>. У меня сердце уходит в пятки от этого самого <<Чпока>>. <<Надо было обноситься по берегу!>>,\mdash промелькивает в голове. Думаю, погнулись кости где\sdash то\ldots~как выяснится позже\mdash лопнул второй шпангоут! С кормы мне казалось, что~там глубже, а коварные камни почти сразу под водной гладью. Наверняка поэтому некоторые авторы туристической литературы советуют Капитану сидеть в байдарке спереди и <<читать>> воду\ldots~правда при этом Рулевой на корме тоже должен быть опытным, что, понятно, не наш вариант.

Ещё налетаем на валун\mdash совершенно озверевший поток носит нас по острым камням, слышен противный шелест по~днищу и~тут я~замечаю, что воды под ногами прибавляется. Я~уже на взводе\mdash чертовски неумело проходим перекаты. Сплошной~шкуродёр! Управляется трёшка из ряда вон плохо, о чем я читал, но~не~верил до~конца\mdash ведь <<Таймень>>\sdash двушка выправляется на~раз\sdash два! С~трёшкой всё оказалось иначе. Мы с Лёней дико лосячим веслами, пытаясь исправить положение, но~нет\mdash снова очередной перекат берёт своё, и~мы получаем ощутимые толчки в днище, да такие, что сердце замирает. Почему~мы не~обнеслись по берегу? Вероятно, всё случилось молниеносно, да и картина на воде при подходе, в силу неопытности, не вызывала опасений.

Опыта прохождения таких препятствий, тем более на байдарке, у меня нет. Об экипаже моём и говорить не проходится\mdash они вообще в первый раз на реке. В~голове промелькивают обрывки разговора со школьными товарищами, которые отдали мне кости\ldots~что\sdash то там про <<отрицаловку>>, ага! Не сразу соображаем гасить скорость перед перекатами, а то и вообще проходить их с отрицательной скоростью относительно воды\mdash <<на~отрицаловке>>, но и это иногда не помогает, если требуется стремительный S\sdash образный манёвр\mdash чрезвычайно трудно быстро развернуть трёшку. Почему, никак не понимаю\mdash на~двушке с женой всё было прекрасно. Мы легко выправляли байдарку на S\sdash образное прохождение препятствий вдвоём. Уже сейчас, после сплава, я понимаю, что Лёню, скорее всего, следовало бы посадить в байдарке на нос, а Аню посередине, а не наоборот. И тогда бы основная гребущая сила\mdash два мужика\mdash оказалась бы на носу и корме. Возможно, с таким распределением сил было бы легче выправлять байдарку. Но, как говорится, наш человек задним умом крепок. В~результате имели то, что имели. Это и называется накоплением опыта, на своих же, правда, ошибках.

Так вот, пройдя очередную шиверу, прилично чирканув днищем и получив снова несколько толчков под кильсон в районе грузового отсека, я обнаруживаю, что воды под ногами прибавляется уж как\sdash то больно стремительно. Не гребём временно, наблюдаю за водой под ногами\ldots~прибывает! ПРОБИЛИСЬ!!! Причём, видимо, очень капитально! Сотрясают воздух крепкие выражения. Достаю кружку, начинаю отчёрпываться. Вода начинает поступать с нешуточной скоростью, отчерпаться не получается! Пробоины не вижу, так бы хоть пальцем заткнул что~ли. Судорожно соображаю что делать. Впереди удачно замечаю песчаную косу за островком\mdash немедленно командую бригаде править туда. Сам продолжаю отчёрпываться кружечкой, чтобы хотя бы кильсон под ногами показался из воды. Добившись этого, бросаю отчерпываться и тоже активнейшим образом включаюсь в греблю. Хорошо, что заветная кружка была близко в кармане.

Влетаем на скорости носом и левым бортом на песчаную косу, вылезаем. Закатав рукава штормовки, обследую днище байдарки прямо на воде, не разгружаясь, чтобы оценить масштаб бедствия. Одна пробоина, ещё дырочка\ldots~вот ещё какая\sdash то шероховатость\ldots~ну, думаю, заткнём их чем\sdash нибудь и пойдём дальше, а в голове крутится мысль\mdash откуда тогда столько воды? Веду рукой дальше по днищу и~тут\ldots~аккурат три пальца входят в пробоину! Шок! Это~просто КАТАСТРОФА, подумалось мне, правда, в более крепких выражениях. Принимаю решение разгружаться\ldots~

Через минут 10, когда все вещи выгружены на~песочек, переворачиваем с Лёней байдарку и\ldots~я~замираю в~ужасе. Леи~целы, но продраны капитально! Если бы их не было\mdash уже бы потопли к чертям. Далее сходу насчитываю минимум~6\sdash 7~дырок! Детальный осмотр выявляет 10~пробоин\ldots~ДЕСЯТЬ! Настроение подавленное. Надо~клеиться. Аня, похоже, ещё не до конца поняла трагизм момента и счастливо фотографируется на фоне окружающей~природы\ldots~

Мы в абсолютнейшей глуши\mdash одна из глухих частей маршрута. Стоим на песчаной косе маленького островка, рядом лежит пробитая байдарка и куча наших вещей. Мысли роем носятся в моей голове, пытаясь сложить картину\mdash как же так получилось?! Могли или не могли мы обнестись? Нет,~не~могли\mdash берега слишком заросшие. Могли лишь погасить скорость, но отчего\sdash то промедлили. И теперь мы имеем что имеем. Но разве не этого я~хотел? Хотел~полнейшего отчуждения от цивилизации? На, получай! Хотел один на один со стихией? Получи, распишись! Ах\sdash ха\sdash ха! Но я не унываю. Наоборот, очнувшись и воспрянув духом, начинаю командовать\mdash надо ремонтироваться. 

Даю разнарядку Лёне добыть дров, чтобы развести костер, а сам тем временем мою днище от песочка\mdash пусть пока просушится\mdash и иду доставать заплатки с~клеем. Хорошо, что мужик, у которого я брал шкуру байдарки, презентовал запасной пузырёк клея! Вспоминаю, что изначально я вообще отчего\sdash то не хотел брать с собой клей, и у меня по спине проходит ледяной холодок\ldots~даже не представляю, как бы мы выбирались из этой глуши, не окажись у нас нормального ремкомплекта. Я был в~полнейшей уверенности, что пробоин эта река нам не сулит. Кроме~того, за три своих предыдущих сплава я ни разу не~пробивался, что и сыграло злую шутку\mdash я был уверен, что и на этот раз всё будет хорошо.

Лёня, тем временем, приносит дров, для чего ему приходится вброд пересекать протоку между нашим островком и, собственно, берегом реки. Разводим кое\sdash как на~песчаной косе костерок, греем воду в~пехотном котелке. Нам нужен кипяток, чтобы прогреть место заплатки. Технология ремонта ПВХ шкуры проста\mdash наносится клей на шкуру, прикладывается заплатка, отводится заплатка, выжидается пара минут, далее заплатка снова прикладывается, плотно прижимается и сверху ставится та самая знаменитая железная кружечка, в которую налит кипяток. Этой кружечкой, как утюжком, проглаживается место заплатки для ускорения полимеризации клея. Большинство пробоин небольшие, заклеиваем их достаточно быстро. Долго ждать кипятка\mdash сильный ветер задувает маленький костерок, плюс сначала зачем\sdash то греем полный котелок воды, и лишь потом сообразим греть на донышке\mdash на~1\sdash 2~кружечки\mdash так быстрее закипает\ldots~

Опасения вызывают две большие пробоины\mdash как раз те, в которые входят по 3 пальца. Виноват я в них, конечно, более всех. Пробоины эти от мест, где к шкуре прилегали колёса тележки для транспортировки упакованной байдарки\mdash я положил эту тележку на самое дно грузового отсека, не учтя того, что сверху лежит гораздо больше вещей, чем когда мы с женой ходили по Киржачу. Тогда я тоже положил тележку в грузовой отсек. Ладно,~теперь учтём! Тележку привязываю резиновыми жгутами на корму байдарки прямо сверху. 

Заклеили все прорехи, выждали время. Аня вовремя подсуетилась с бутербродами\mdash перекусили, передохнули, успокоились. У Ани был термос с заваренным шиповником, и он пришёлся как нельзя кстати. Потом спустили байдарку на~воду для проверки. Оказалось, что одна заплатка\mdash как раз от колеса тележки\mdash фильтрует! Решаю на эти две большие пробоины поставить заплатки ещё и изнутри шкуры, а также промазать края заплаток клеем. Сказано\mdash сделано, и ещё через полчаса всё готово. Проверка на воде\mdash фильтрации нет, можно идти. В общей сложности потеряли на заклейку 4 часа. Да\sdash а\sdash а, выбиваемся из графика! 

В этот день больше серьёзных перекатов не~встретилось, дошли до разрушенного деревянного моста близ заброшенного ж/д остановочного пункта Тургошь. Место~на~правом берегу мне сразу приглянулось, и мы встали на ночёвку, тем более что времени уже было к восьми вечера. Координаты стоянки\mdash N~59\degree~19.660$^\prime$~ E~35\degree~10.407$^\prime$. Позже, в Москве, по треку навигатора я точно посмотрел, что, несмотря на ремонтную остановку в 4 часа, мы прошли за этот ходовой день 24 километра. Очень~солидно! Но~я~планировал больше, и теперь все мои прикидки по~километражу и прохождению маршрута сбились.
 
Место на правом берегу, выбранное нами под лагерь, было несколько обжитым и замусоренным. Разбили лагерь. Разведением~костра занялся я, Лёня\mdash палаткой, а Аня ужином\mdash будут опять макароны с тушёнкой\mdash нет сил соображать что\sdash то поинтереснее. Тем временем подчистили стоянку, сожгли мусор в костре. Быстро стемнело. Я~развёл 96\sdash й, и мы сели вкушать приготовленные яства уже в~глубоких сумерках.

Около Тургоши располагается пионерлагерь, и музыка оттуда не стихала до 11 вечера, дискотека. Заманчиво было бы сходить, но далековато, да и устали сильно. Сидим у костра, я прокручиваю в голове пережитое. Вот это выдался денёк! 10 пробоин\mdash не шутки! Для меня\sdash то это вновь, а ребята вообще первый раз на реке\mdash вот, думаю, выпало им на первый сплав испытаний. Но мы всё преодолели, сохранили боевой настрой и даже почти наверстали километраж благодаря сильному течению. Горько за пробитую шкуру, конечно. Пусть это будет нашей жертвой богам леса и реки.

Август\mdash пора метеоров. Жадно вглядываемся в~безоблачное ночное небо, силясь увидеть хоть один. Красота природы безгранична, думается мне. А человек\mdash не царь её, но составная часть. Ещё точнее, он\mdash варвар, несущий разрушения во имя выживания, а больше вопреки, особенно в последнее время. Так,~лёжа на коврике у костра под звёздным небом, размышляя о далёких галактиках и~попыхивая трубочкой, мне становится необычайно хорошо от осознания происходящего с нами. Я~радуюсь тому, что прошли плотину сегодня, что не потопили байдарку на перекате, что не утонули, получив пробоину; просто упиваюсь восторгом от нашего похода\mdash наслаждаюсь природой, потрескиванием дров, ароматом костерка\ldots~и~ещё более жду неизвестности, которая поджидает нас~на~маршруте.

Увидели\sdash таки с Лёней один метеор в полночном небе. Или~это следствие 96\sdash го? Всё может быть. Вокруг нас стоит тихая звёздная ночь, чуть слышно плещется Лидь у~разрушенного моста, а~мы сидим у костра, задрав головы к~бриллиантовому небу. Красота! Звёзды яркие\sdash яркие. Где\sdash то~там, далеко над нами, скорее всего, есть иные цивилизации\mdash не может ведь быть так, что мы одиноки во~Вселенной\ldots~они даже могут слать нам какие\sdash то свои сигналы или позывные, но\ldots~Земля~ещё не готова к этому. 

Мы~не живём в гармонии\mdash ни с природой, ни~друг с другом в планетарных масштабах\ldots~и всё в нашем посткапиталистическом мире подчинено одному\mdash деньгам. Какие~уж тут иные цивилизации и всё вот это вот\mdash в~своей бы разобраться для начала. Чтобы~не~хотелось от~неё убегать в~неведомые дали в~слепом желании забыться\ldots~Так~называемая Эра Разобщённого Мира \cite{ТуманностьАндромеды}, чтоб~её. Будет~ли~у~нас Эра~Великого~Кольца когда\sdash нибудь?

Усугубив, спать завалились глубоко за полночь, и~я~моментально провалился в~глубокий и~крепкий молодецкий сон, опять~напрочь забыв про~судовой~журнал.

\begin{center}
	\psvectorian[scale=0.4]{88} % Красивый вензелёк :)
\end{center}