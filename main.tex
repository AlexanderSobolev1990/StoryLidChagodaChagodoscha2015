%%%%%%%%%%%%%%%%%%%%%%%%%%%%%%%%%%%%%%%%%%%%%%%%%%%%%%%%%%%%%%%%%%%%%%%%%%%%%%%%
%
% \file       main.tex
% \brief      Главный файл настроек TeX проекта
% \date       24.08.22 - создан
% \author     Соболев А.А.
% \details    Для сборки в TexStudio - выбрать компилятор xelatex в настройках проекта
%             Для сборки при помощи CMake - mkdir build && cd build && cmake .. && make	
%
%%%%%%%%%%%%%%%%%%%%%%%%%%%%%%%%%%%%%%%%%%%%%%%%%%%%%%%%%%%%%%%%%%%%%%%%%%%%%%%%
\listfiles
\documentclass[headings=chapterprefix,a5paper]{book}
\usepackage[monochrome]{color}
\usepackage[12pt]{extsizes}
\usepackage[left=1.5cm,right=1.5cm,top=2.0cm,bottom=2.0cm]{geometry}
\linespread{1.0} % по умолчанию 1.0
%
\usepackage{tikz}
\usepackage{tocloft}
\usepackage{pgfornament}
\usepackage{amsmath}
\usepackage{gensymb} % Для знака градуса
\usepackage{cite}
\usepackage{pstricks}
\usepackage{psvectorian}
\usepackage{fontspec}
\usepackage{pdfpages}
\usepackage[russian]{babel}

\setmainfont[%
ItalicFont=NewCM10-Italic.otf,%
BoldFont=NewCM10-Bold.otf,%
BoldItalicFont=NewCM10-BoldItalic.otf,%
SmallCapsFeatures={Numbers=OldStyle}]{NewCM10-Regular.otf}

\setsansfont[%
ItalicFont=NewCMSans10-Oblique.otf,%
BoldFont=NewCMSans10-Bold.otf,%
BoldItalicFont=NewCMSans10-BoldOblique.otf,%
SmallCapsFeatures={Numbers=OldStyle}]{NewCMSans10-Regular.otf}

\setmonofont[ItalicFont=NewCMMono10-Italic.otf,%
BoldFont=NewCMMono10-Bold.otf,%
BoldItalicFont=NewCMMono10-BoldOblique.otf,%
SmallCapsFeatures={Numbers=OldStyle}]{NewCMMono10-Regular.otf}

\usepackage{afterpage}
%\usepackage[x-1a3]{pdfx} % Вызывает невозможность собирать в texstudio!
\usepackage{ctable}
\usepackage{longtable}
\usepackage{graphicx}
\graphicspath{ {./images/} }
%--------------------------------------
% эпиграф
\usepackage{epigraph}
\setlength{\epigraphwidth}{0.60\textwidth}%0.65
\renewcommand{\textflush}{flushleft} \renewcommand{\sourceflush}{flushleft}
\let\originalepigraph\epigraph 
\renewcommand\epigraph[2]{\originalepigraph{\textit{#1}}{\scriptsize{#2}}} %\textsc

%------------------------------------------------------------------------------------------------------------
%настройки для А4
%\usepackage[12pt]{extsizes}
%\usepackage[left=2.5cm,right=2.5cm,top=2.5cm,bottom=2.5cm]{geometry}
%\linespread{1.15} % по умолчанию 1.0
%------------------------------------------------------------------------------------------------------------
%настройки для А5
%\usepackage[11pt]{extsizes}
%\usepackage[left=1.5cm,right=1.5cm,top=2.0cm,bottom=2.0cm]{geometry}
%\linespread{1.0} % по умолчанию 1.0
%------------------------------------------------------------------------------------------------------------
%настройки для А4
%\newcommand{\corner}[1]{%
%	\begin{tikzpicture}[remember picture, overlay]
%	\node[anchor=north east, shift={(-2.5cm,-5.3cm)}] at (current page.north east){%
%		\pgfornament[width=2.2cm]{#1}};
%	\end{tikzpicture}%
%}
%------------------------------------------------------------------------------------------------------------
%настройки для А5
\newcommand{\corner}[1]{%
	\begin{tikzpicture}[remember picture, overlay]
	\node[anchor=north east, shift={(-1.5cm,-4.2cm)}] at (current page.north east){%
		\pgfornament[width=2.2cm]{#1}};
	\end{tikzpicture}%
}
%------------------------------------------------------------------------------------------------------------
\usepackage{indentfirst} % Первая строка главы - с красной строки
\setlength{\parindent}{1.0cm} % Отступ слева первой абзаца
\setlength{\parskip}{0.25cm} % Отступ между абзацами

% Заголовки сверху и снизу каждой страницы (Headers & footers)
\usepackage{fancyhdr}
\pagestyle{fancyplain}

\fancyhead[LE]{\fancyplain{}{}}
\fancyhead[CE]{\fancyplain{}{}}
\fancyhead[RE]{\fancyplain{}{}}

\fancyhead[LO]{\fancyplain{}{\bfseries\rightmark}}
\fancyhead[CO]{\fancyplain{}{}}
\fancyhead[RO]{\fancyplain{}{}}

\fancyfoot[LE]{\fancyplain{}{\bfseries\thepage}}
\fancyfoot[CE]{\fancyplain{}{}}
\fancyfoot[RE]{\fancyplain{}{\bfseries\scriptsize \MyVarBookName}}

\fancyfoot[LO]{\fancyplain{}{\bfseries\scriptsize \MyVarAuthorName }}
\fancyfoot[CO]{\fancyplain{}{}}
\fancyfoot[RO]{\fancyplain{}{\bfseries\thepage}}

\renewcommand{\footrulewidth}{0.4pt}
\renewcommand{\chaptermark}[1]{%
	\markboth{#1}{}%
}

\renewcommand{\chaptermark}[1]{\markright{\chaptername\ \thechapter.\ #1}{}}

% Пустая страница
\newcommand\blankpage{%
	\null
	\thispagestyle{empty}
	\newpage
}

\newcommand{\sdash}{\nobreakdash-}  % Дефис неразрывный без пробелов до и после
\newcommand{\ndash}{\nobreakdash~--~}  % Короткое тире неразрывное с пробелами до и после
\newcommand{\mdash}{\nobreakdash~---~} % Длинное тире неразрывное с пробелами до и после

\renewcommand{\cftchappresnum}{Глава }
\setlength{\cftbeforechapskip}{0.7em}
%\renewcommand\cftchapafterpnum{\vskip 0em} % for spacing after each entry
\AtBeginDocument{\addtolength\cftchapnumwidth{\widthof{\bfseries Глава~ }}}

\renewcommand\cftchapdotsep{\cftdotsep} % Добавление точечек к элементу chapter

\newcommand\MyVarAuthorName{А.А.~Соболев}
\newcommand\MyVarBookName{Повесть о сплаве на байдарке}
\newcommand\MyVarBookNamesec{по рекам Лидь, Чагода, Чагодоща}

\newcommand{\UDK}{821.161.1}
\newcommand{\BBK}{84 (2Рос-Рус) 6}
\newcommand{\BibCode}{C54}
\newcommand{\ISBN}{ISBN 978-5-6048796-4-1}

\renewcommand*{\thefootnote}{\fnsymbol{footnote}}
\usepackage[symbol]{footmisc}

\newcommand{\vepsianrose}{%
	\begin{figure}[h!]
%		\vspace*{-77mm}
		\vspace*{-70mm}
%		\hspace*{8.15cm}\includegraphics[scale=0.1]{vepsroseb} % Черный
		\hspace*{8.8cm}\includegraphics[scale=0.1]{vepsroseb} % Черный
	\end{figure}
	\vspace*{4.0cm}
}

\usepackage{titlesec}
\usepackage[colorlinks=true,linktoc=all]{hyperref} % hyperfootnotes=false - не выделять сноски
\usepackage{bookmark}

\begin{document}
\input{coordsvalues}
\relpenalty=10000
\binoppenalty=10000
\clubpenalty=10000  % Это костыль против
\widowpenalty=10000 % "висячих" строк

\righthyphenmin=200 % Избавляемся
{\sloppy
\includepdf{обложка_а5.pdf}
\newpage
\null
\thispagestyle{empty}
\newpage
\begin{titlepage}
	\newpage
	\begin{center}
		\huge \textbf \MyVarAuthorName
	\end{center}	
	\vspace{3cm}	
	
	\begin{center}
	\begin{tikzpicture}
	\fill[fill=black!30!green,] (0,0) rectangle (10.8,0.2);
	\end{tikzpicture}
	\end{center}
	
	\begin{center}
		\huge \textbf {ПОВЕСТЬ О СПЛАВЕ\\НА БАЙДАРКЕ}
	\end{center}	
	\vspace{0.0cm}

	\begin{center}
	\begin{tikzpicture}
	\fill[fill=black!30!green,] (0,0) rectangle (10.8,0.2);
	\end{tikzpicture}
	\end{center}

	\begin{center}
		\LARGE {по рекам Лидь, Чагода, Чагодоща}
	\end{center}

	\begin{center}
	\begin{tikzpicture}
	\fill[fill=black!30!green,] (0,0) rectangle (10.8,0.2);
	\end{tikzpicture}
	\end{center}
		
	\vspace{\fill}	
	\begin{center}
		\normalsize
		Москва \linebreak 
		2022 \linebreak
		Самиздат
	\end{center}	
\end{titlepage}

%\chapter*{}
%\newpage
{
\thispagestyle{empty}
%
\small{
\begin{flushleft}
\textbf{%
	УДК\UDK \\
	ББК\BBK \\
	\BibCode \\
}
\end{flushleft}
%
\vspace{3cm}
%
%\begin{flushleft}
%{
\begin{tabular}[c]{>{\raggedright}m{14mm} >{\raggedright}m{95mm} }
	\textbf{\BibCode} & \MyVarAuthorName \tabularnewline
	~ & \MyVarBookName \tabularnewline
	~ & М.:Маска,2022\mdash 128 c. \tabularnewline
	~ & \textbf{\ISBN}
\end{tabular}
%}
%\end{flushleft}
%
\vspace{5.5cm}
%
\begin{flushright}
\textbf{%
	УДК\UDK \\
	ББК\BBK \\
	\BibCode \\
}
\end{flushright}
%
%\vspace{\fill}
\vspace{1cm}
%
\begin{longtable}[c]{>{\raggedright}m{55mm} >{\raggedleft}m{55mm} }
	\textbf{\ISBN} & {\copyright~\MyVarAuthorName,~2022} \tabularnewline
\end{longtable}
}
}
%\afterpage{\blankpage}
\input{лидь2015/аннотация}
%\afterpage{\blankpage}
% Замена \blankpage в данном случае:
\newpage
\null
\thispagestyle{empty}
\newpage
%
{
%\setlength{\cftbeforetoctitleskip}{1.5cm}%чтобы влезло на 1 страницу 11 шрифт - 15 глав
%\setlength{\cftbeforetoctitleskip}{0.5cm}%чтобы влезло на 1 страницу 12 шрифт - 15 глав
%\setlength{\cftbeforetoctitleskip}{-0.5cm}%чтобы влезло на 1 страницу 12 шрифт - 15 глав + Литература
\setlength{\cftbeforetoctitleskip}{-0.1cm}%чтобы влезло на 1 страницу 12 шрифт - 15 глав + Литература + Слово "Глава"
\pdfbookmark[0]{\contentsname}{toc} % Добавлять перед \tableofcontents
\tableofcontents
}
\afterpage{\blankpage}
\input{лидь2015/введение}
\afterpage{\blankpage}
%
\input{лидь2015/несколько_слов_о_сплавах}
\input{лидь2015/байдарка} 
\input{лидь2015/маршрут_и_сборы} 
\input{лидь2015/отъезд}
\input{лидь2015/заброска}
\input{лидь2015/катастрофа}
\input{лидь2015/фонтан}
\input{лидь2015/первая_дневка}
\input{лидь2015/дозаправка_в_чагоде}
\input{лидь2015/долгий_переход}
\input{лидь2015/кабаны_и_удача}
\input{лидь2015/вторая_дневка}
\input{лидь2015/досрочный_антистапель}
\input{лидь2015/возвращение}
\input{лидь2015/заключение}
%
\bibliographystyle{abbrvnatmy.bst}%\bibliographystyle{apalike}%{plain}%{alpha}%
\bibliography{bibliography}
\addcontentsline{toc}{chapter}{Литература}
{
\newpage
\thispagestyle{empty}
\begin{center}
{\small Литературно\sdash художественное издание}\\
\vspace{1.6cm}
{\Large \MyVarAuthorName}\\
\vspace{1.6cm}
{\Large\textbf\MyVarBookName}\\
\vspace{0.4cm}
{\Large\textbf\MyVarBookNamesec}\\
\vspace{1.0cm}
{\small%
Текст публикуется в авторской редакции\\
\vspace{1.0cm}
Обложка: инвертированное по горизонтали фото обрыва\\
на р.Чагодоща, координаты места съёмки:\\N~59.19222\degree~E~36.23360\degree,\\используется с разрешения автора фото,\\ Красавина С.Ю., 2016~г.\\
\vspace{1.5cm}
Сдано в набор 14.09.2022.\\
Гарнитура New Computer Modern.\\
Формат А5. Бумага офсетная.\\
Тираж 20 экземпляров. Заказ \number 442.\\
\vspace{1.0cm}
Отпечатано в ООО~<<ИПЦ~"`Маска"'>>\\
Москва, ул. Малая Юшуньская, д.~1, корп.~1.\\
Тел. (495) 510-32-98\\
www.maska-print.ru
}
\end{center}
}
%\afterpage{\blankpage} % чтобы число страниц было кратно 8
}
\end{document}